% ----------------------------------------------------------
% Introdução 
% Capítulo sem numeração, mas presente no Sumário
% ----------------------------------------------------------

\chapter*[Introdução]{Introdução}
\addcontentsline{toc}{chapter}{Introdução}

Este relatório descreve o método utilizado para o projeto de um sistema de rádio enlace ponto a ponto, com base nos dois pontos fornecidos. O enlace de rádio pode ser definido como : "
Uma aplicação da transmissão de informação por meio de ondas eletromagnéticas, se caracterizando como uma das aplicação que faz parte das Segundo Tude , “Um enlace rádio digital ponto a ponto é utilizado para o transporte de informação entre dois pontos fixos,tendo o espaço livre como meio de transmissão (wireless)”.[1]

\section*{Motivação}\label{sec:motivacao}
\addcontentsline{toc}{section}{Motivação}

\lipsum[35]

\section*{Objetivos}\label{sec:objetivos}
\addcontentsline{toc}{section}{Objetivos}

Como objetivo geral este trabalho propõe...

Para isso, objetivos específicos foram assim definidos:

\begin{itemize}
\item especificar objetivos específicos
\item listar objetivos específicos
\end{itemize}

\section*{Organização do documento}\label{sec:objetivos}
\addcontentsline{toc}{section}{Organização do documento}

Parte I, Parte II, Parte III... \lipsum[32]
